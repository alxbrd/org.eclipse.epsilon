\subsubsection{Moving Model Elements into a Different Package}
\label{sec:MoveToPackage}

A common refactoring when modelling in UML is to move model elements, particularly Classes, between different packages. When moving a pair of classes from one package to another, the associations that connect them must also be moved in the target package. To automate this process, Listing \ref{lst:MoveToPackage} presents the \texttt{MoveToPackage} wizard.

\begin{lstlisting}[float=tbp, 
	basicstyle=\ttfamily\footnotesize, 
	flexiblecolumns=true, 
	numbers=none, 
	nolol=true, 
	caption=Implementation of the MoveToPackage Wizard, 
	label=lst:MoveToPackage, 
	numbers=left, 
	language=EWL, 
	tabsize=2
]
wizard MoveToPackage {
	
	// The wizard applies when a Collection of
	// elements, including at least one Package
	// is selected
	guard { 
		var moveTo : Package;
		if (self.isKindOf(Collection)) {
			moveTo = self.select(e|e.isKindOf(Package)).last();
		}
		return moveTo.isDefined();
	}
	
	title : "Move " + (self.size()-1) + " elements to " + moveTo.name
	
	do {
		// Move the selected Model Elements to the
		// target package
		for (me in self.excluding(moveTo)) {
			me.namespace = moveTo;
		}
		
		// Move the Associations connecting any
		// selected Classes to the target package
		for (a in Association.allInstances) {
			if (a.connection.forAll(c|self.includes(c.participant))){
				a.namespace = moveTo;
			}
		}
	}
	
}
\end{lstlisting}

The wizard applies when more than one element is selected and at least one of the elements is a \emph{Package}. If more than one package is selected, the last one is considered as the target package to which the rest of the selected elements will be moved. This is specified in the \emph{guard} part of the wizard.

To reduce user confusion in identifying the package to which the elements will be moved, the name of the target package appears in the title of the wizard. This example shows the importance of the decision to express the title as a dynamically calculated expression (as opposed to a static string). It is worth noting that in the \emph{title} part of the wizard (line 14), the \emph{moveTo} variable declared in the \emph{guard} (line 7) is referenced. Through experimenting with a number of wizards, it has been noticed that in complex wizards repeated calculations need to be performed in the \emph{guard}, \emph{title} and \emph{do} parts of the wizard. To eliminate this duplication, the scope of variables defined in the \emph{guard} part has been extended so that they are also accessible from the \emph{title} and \emph{do} part of the wizard.
