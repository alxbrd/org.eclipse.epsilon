\chapter{Introduction}
\label{chp:Introduction}

The purpose of this book is to provide a complete reference of the languages provided by the Epsilon project (\url{http://www.eclipse.org/gmt/epsilon}). The first version of this book is based on a cut-down version of the first author's PhD thesis so its style may be a bit different than other books on programming languages you may have read so far - but this will hopefully improve in future versions.

\section{What is Epsilon?}

Epsilon, standing for Extensible Platform of Integrated Languages for mOdel maNagement, is - as it's extended name hints - a platform for building consistent and interoperable task-specific languages for model management tasks such as model transformation, code generation, model comparison, merging, refactoring and validation.

Epsilon currently provides the following languages:

\begin{itemize}
	\item Epsilon Object Language (EOL)
	\item Epsilon Validation Language (EVL)
	\item Epsilon Transformation Language (ETL)
	\item Epsilon Comparison Language (ECL)
	\item Epsilon Merging Language (EML)
	\item Epsilon Wizard Language (EWL)
	\item Epsilon Generation Language (EGL)
\end{itemize}

For each language Epsilon provides Eclipse-based development tools and an interpreter\footnote{The interpreters are not bound in any way with Eclipse and can also be used in standalone Java applications.} that can execute programs written in this language. Epsilon also provides a set of ANT tasks for creating workflows of different tasks (e.g. a validation followed by a transformation followed by code generation). The following chapters present the syntax of each language and a few usage examples.

\section{How To Read This Book}

If you are reading this book, there are good chances that you are already interested in using a particular task-specific language provided by Epsilon (e.g. EVL for model validation or EWL for refactoring). In this case, you don't have to need to read about all the languages: you first need to spend some time reading Chapter \ref{chp:InstallingEpsilon} in order to find out how to install Epsilon and Chapter \ref{sec:Design.EOL} that presents the core Epsilon Object Language (EOL), as all languages of the platform extend EOL both syntactically and semantically. Then you can proceed to the chapter that discusses the particular language you are interested in (e.g. Chapter \ref{sec:EVL} for EVL).

\section{Questions and Feedback}

Our intention is to keep this book a live project that will evolve in parallel with the evolution of Epsilon. Therefore, your feedback on any omissions, errors or outdated content is critical and much appreciated (and also you will win a place for your name in the Acknowledgements section of the book :D ). Please send your feedback to the Eclipse eclipse.epsilon newsgroup under news.eclipse.org. If you are not sure about how to connect and post to the newsgroup please refer to \url{http://wiki.eclipse.org/index.php/Webmaster_FAQ#How_do_I_access_the_Eclipse_newsgroups.3F} for detailed instructions.

\section{Additional Resources}

As mentioned above, information about Epsilon and examples are available in many different places. If you can't find what you are looking for in this book there are a few other places where you may try.

\subsection{Epsilon Eclipse GMT}
Epsilon is a component of the Eclipse Modelling GMT project and it is hosted in \url{http://www.eclipse.org/gmt/epsilon}. In the documentation section \url{http://www.eclipse.org/gmt/epsilon/doc} there is documentation about several features of Epsilon, and the Cinema \url{http://www.eclipse.org/gmt/epsilon/cinema} contains a number of Flash screencasts that demonstrate different languages and tools of Epsilon in action.

\subsection{EpsilonLabs}

EpsilonLabs is a satellite project of Epsilon that hosts experimental applications/extensions of Epsilon or other content that cannot be shared under Eclipse.org due to licensing issues (e.g. incompatibility with EPL). EpsilonLabs is located in \url{http://epsilonlabs.sf.net}

\subsection{Epsilon Weblog}
In November 2007 we started a blog where we've been reporting new applications and extensions of Epsilon. The blog provides the latest information on the project and is located in \url{http://epsilonblog.wordpress.com}